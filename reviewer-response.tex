\documentclass{article}

%#!platex ieice.answer.tex && dvipdfmx -p a4 ieice.answer
%# !pdflatex -shell-escape ieice.answer.tex
%#BIBTEX jbibtex ieice.answer
%#LPR open ieice.answer.pdf

%\usepackage[pdftex,breaklinks,pdfborder={0 0 0}]{hyperref}
\usepackage[dvipdfm,breaklinks,pdfborder={0 0 0}]{hyperref}
\usepackage{booktabs}
\usepackage{xspace}
\usepackage{amsmath}	% required for `\equation*' (yatex added)

\renewcommand{\baselinestretch}{1.2}
\newcommand{\mailaddr}[1]{$<$\href{mailto:#1}{\texttt{#1}}$>$}
\newcommand{\comment}{\subsection{Comment}\em}
\newcommand{\subcomment}{\subsubsection{Comment}\em}
\newcommand{\answer}{\rm \subsubsection*{Answer}}
\newcommand{\quotepaper}[1]{%
  \begin{center}
    \framebox[.95\linewidth]{\begin{minipage}{.9\linewidth} #1 \end{minipage}}
  \end{center}}
\newcommand{\ellipsis}{[\thinspace\ldots]\xspace}

\newcommand{\dsetone}{\mbox{\texttt{2008-12-01}}\xspace}
\newcommand{\dsettwo}{\mbox{\texttt{2008-12-19}}\xspace}
\newcommand{\dsetsigcomm}{\mbox{\texttt{sigcomm2004}}\xspace}

%%%%%NEWCOMMANDS
\newcommand{\ket}[1]{\ensuremath{|#1 \rangle}}
\newcommand{\bra}[1]{\ensuremath{\langle #1|}}
\newcommand{\braket}[2]{\ensuremath{\langle #1|#2 \rangle}}
\newcommand{\ketbra}[2]{\ensuremath{|#1 \rangle \langle #2|}}
\newcommand{\ro}[1]{\ensuremath{|#1 \rangle \langle #1|}}
\newcommand{\av}[1]{\ensuremath{\langle #1 \rangle}}
\newcommand{\real}{\ensuremath{\mathrm{Re}}}
\newcommand{\trace}{\ensuremath{\textsf{Tr}}}
%%%%%

\title{Summary of Changes Introduced \mbox{According to Reviewers' Comments}}

\begin{document}

\maketitle

\begin{description}
   \item[Journal:] Communications of the ACM
   \item[Manuscript ID:] CACM-12-04-1344
   \item[Original paper title:] State of the Art in Quantum Computer Architectures
   \item[Revised paper title:] How to Design a Quantum Computer
   \item[Authors:] Rodney Van Meter and Clare Horsman
   \item[Corresponding author:] Rodney Van Meter \mailaddr{rdv@sfc.wide.ad.jp}
   \item[Affiliation:] Keio University, Japan.
\end{description}


\bigskip\noindent Dear Editor,

\medskip 

We have addressed all of the reviewers' extensive (and very helpful)
comments. As suggested, herewith we submit a new version of the paper.

We are submitting two documents: the article itself, and this
document, detailing the changes.  Ordinarily, we would include a
``diff'' version marking modifications since the original submission,
but the extensive reorganization of the paper would limit its
usefulness.

Below, please find the detailed list of changes, together with the
reviewers' comments.  This letter ends with a list of additional
modifications and a proposed list of additional references beyond
those currently included in the article.

\medskip\noindent Best regards,

\begin{list}{}{\setlength{\itemsep}{0mm}}
\item Rodney Van Meter and Clare Horsman
\end{list}

\cleardoublepage

\section{Overview of Revision}

\subsection{Reorganization of the Article}
\label{sec:rearch}

Following the suggestion of the Associate Editor, and with specific
advice from ``friendly'' reviewers from the classical computing
community (i.e., CACM's regular audience), we have substantially
reorganized the article.  Figure 2, showing the relationships among
some sub-fields of study that all must contribute to quantum computer
architecture, has been slightly revised and is now used as the pivot
around which the article is organized.

The five ``layers'' in the figure are discussed in four sections.
Each section presents the issues in general terms, then briefly
discusses the current state of the art and open research area in an
integrated fashion, rather than having a standalone ``Open Problems''
section toward the end of the article.

The order of sections begins at the bottom and ends at the middle,
where overall system structure comes in.

The ``Applications'' section has been renamed ``Workloads''.  It
includes programming tools, and remains the shortest section of the
article.  Prof. Aho requested more in this section, but there is
simply not room within the space limit.  This area quite readily
warrants a separate review/tutorial of its own.

Overall, this structure for the paper seems to offer more integrated
flow, ending with the discussion of the scaling of one architecture to
give readers an understanding of the macro-level system architecture
issues.

\subsection{Sidebar}

The sidebar on quantum computation (now attached as an appendix,
formerly Section 2) has been shortened and revised to focus on
gate-based computation and adiabatic computation, the only two forms
of quantum computation discussed in this article.

\subsection{D-Wave}
\label{sec:d-wave}

The issue of whether, and how, to address D-Wave generates discussion
with everyone who reads the article, so let us take a moment to
explain our reasoning.  D-Wave has certainly made few friends in the
research community, primarily due to their misleading claims to be
able to solve NP-complete problems and the lack, until recently, of
peer-reviewed publications demonstrating what their machine actually
{\em can} do.  Speaking frankly, we are not fans of the company.

Most researchers do not believe that what they have built is a quantum
computer.  It's not even clear that {\em they} believe that what they
have built is a quantum computer.  However, what they are doing is
solving important problems in the {\em engineering} of quantum
computers at moderate scale: improved magnetic shielding, multiple
qubit interconnects, on-chip integration of control circuits, and even
software techniques.  This is more visible in the set of patents they
hold than in the few peer-reviewed papers they have issued.

As perhaps the only publicly-known organization working at this scale,
D-Wave is hard to completely ignore.  Moreover, many CACM readers will
have heard of D-Wave's sale, and leaving it out of the article
altogether will leave readers with unanswered questions.  Our hope is
to place their work in context (controversy and all), in as
even-handed a way as is possible at this juncture.

Reviewer \#1 approves of the discussion of D-Wave in general, though
he or she asks for some changes, which we have made.  Reviewer \#4
disapproves, and while we are sympathetic, we disagree for the reasons
stated here.

\section{Review Co-chair}

Co-Chair, Review Committee: Al Aho\\
Comments to the Author:\\

\comment

A rewritten version of this paper should be published.  It describes
in a sobering way the multitude of problems that need to be solved if
we are going to be able to build practical, scalable quantum computers
using currently known technologies.  Many researchers working in
quantum computing are probably not aware of the gamut of daunting
systems and architectural issues that need to be solved before we can
have practical, scalable quantum computers.

\answer

Thank you for the positive comments.  Indeed, our experience in
discussing these issues for nearly a decade is that experimentalists
have not yet focused on the fact that a large system is more than a
larger version of a small system.  Conversely, the CS theorists
working in the area rightfully take great pride and pleasure in
pursuing the fundamental issue of complexity, and take on the critical
task of developing the algorithms we must have, but worry less about
the machines that will ultimately execute these algorithms.  This gap
is where we, and the architecture community in general, fit in.  We
believe this bridge is critical to determining whether quantum
computation is ultimately transformational, or a historical footnote.
Fortunately, both ends of the spectrum are gradually looking toward
the middle, and architectural principles will figure prominently in
the next generation of experimental systems.

\comment

And this paper does not address the critical problem of how such
quantum computers are going to be programmed.

\answer

Indeed, although we include a paragraph on the importance of
programming tools, this is a distinct area in its own right, and would
warrant a separate review.  While we would love to have the
opportunity to write such an article, others such as Prof. Aho's
former student Dr. Svore, Prof. Markov, Dr. Maslov, or one of the
teams in the IARPA program would all be excellent choices as authors.

\comment

Although the title of the paper is ``State of the Art in Quantum
Computer Architectures'', there are no practical, scalable quantum
computers in existence today (leaving aside the D-Wave Systems quantum
computer, about which there is only a lot of speculation as to what
kind of computer it really is). I fully agree with the reviewer who
suggested the paper should be retitled.

\answer

Retitled ``How to Design a Quantum Computer''.

\comment

My major concern with the paper is the one raised by the Associate
Editor -- will the paper in its current form be accessible to the
average CACM reader? I would strongly encourage the authors to rewrite
their paper with this question in mind. In its current form, to
researchers in quantum computing, the paper is an easy-to-read, but
very superficial, overview of current approaches; to readers with no
background in quantum information processing, I am afraid it will
leave most of them completely mystified.

\answer

As described above, we have restructured the paper with the goal of
improving the flow and making key problems easier to grasp.  After
restructuring, we solicited ``friendly'' reviews from five classical
computer systems researchers and incorporated their advice.

\comment

However, there are two reasons why I am encouraging ultimate
publication of a revised version.  One is that such a paper would
dispel a lot of uninformed optimism about the immediate future of
quantum computers. Another, and perhaps more important, is that such a
paper might stimulate some researchers currently not in quantum
information processing to come up with new approaches that will
overcome the daunting problems currently confronting this field.\\
 --al aho

\answer

Indeed, Prof. Aho has identified our major motivations for writing
this paper, and we hope the message comes through.  An additional,
minor goal is to be able to use the open areas outlined in this paper
to encourage researchers in quantum computation to adjust their focus
and build collaborations with the systems researchers whose expertise
is ultimately needed.

\section{Associate Editor}

Associate Editor: Markov, Igor\\
Comments to the Author:\\

\comment
We solicited feedback from four experts in quantum computing with very
different background. Their open comments about this submission and
comments to the Associate Editor paint a mixed picture. Even the most
positive reviewers point out serious issues that need to be
resolved. While it is desirable to have a summary of recent research
(or even a summary of the state of the art) in quantum architectures,
the submission suffers several major problems and has numerous low-
and mid-level deficiencies. In particular, it is not clear if the
(reasonable) constraints found in reviews can be simultaneously
satisfied. At minimum, this would require rewriting the paper, which
is why the authors are advised to resubmit their next revision as
new. To make sure that the draft is accessible to the general
audience, consider soliciting feedback from colleagues not involved in
quantum computing.

\answer

The article has now been thoroughly revised to take into account the
comments of all of the CACM-solicited reviews, as well as five new
reviews we solicited from classical systems researchers after
revision.  All of the mid- and low-level specific issues raised by the
reviewers have been addressed.  The new structure creates a much more
logical flow to the paper as a whole, and the incorporation of the
research areas into each section allows them to be seen more clearly
in context.  The article now covers the importance of architecture,
the key advances of recent architectural research, and the open areas
where both the classical and quantum communities will converge to
create large-scale systems.  We believe the revised version
simultaneously satisfies the constraints on length and accessibility
and the need for comprehensive coverage.

\section{Reviewer: 1}

Comments to the Author
\comment

``State of the Art in Quantum Computer Architectures'' by Rodney Van
Meter and Clare Horsman presents a general overview of current
thinking about how a quantum computer will be architected.  This
review is generally well written, attempting to solve the difficult
problem of giving an overview of the complex and diverse field of
quantum computing, while also discussing the main sub-area quantum
architectures.  The parts of the review that shine are those which
focus on big picture architectural understandings (just pointing out,
for example, that a quantum computer will have huge error correcting
overhead is probably important for an article in CACM).  I think the
paper should be published in CACM, subject to address the few major
points below (I've also added minor points of contention that I hope
the authors will consider.)

\answer

Thank you for the positive comments.

\comment

1) The section on quantum computation should be put in sidebar or
considerably shortened.  This section basically ruins the flow of the
article where it currently is.  Many computer scientists have a basic
understanding now of what a quantum computer is, and further the
sections that follow don't depend heavily on this section.

\answer

The section has been substantially shortened to focus on the
gate-based and adiabatic models that appear in the main text.  Our
intention is for it to be a sidebar to the main article, where it
won't interrupt the flow of the text.  It now appears as an appendix
to this submission.

\comment

2) The section on quantum error correction needs to call out more
explicitly that we can define measures like the rate of the quantum
codes, but that this rate isn't necessarily the most important
property of these codes, and that other features of the architecture
will dictate which codes are used.  The authors are reaching towards
this in this section, but this needs to be stated more clearly.

\answer

This is a good observation.  We added a paragraph in the middle of the
section discussing rate and introducing how a code is selected.

\comment

3) The take home message for the section on quantum experimental
progress is actually the second to last one, in which one discusses
the actual sizes of the devices.  This will come as a surprise to most
readers. Reworking this in such a way that this is more prominent
(making it a table referenced early on), would significantly improve
this section.

\answer

The table in the paper has been revised, in part with this goal in
mind.

\comment


Smaller points of contention / nits.

pg. 2:\\
Fig 1: ``This figure demonstrates the complexity of making pronouncements about the speed of quantum computers''
I'm not sure the figure demonstrates complexity.  But it does demonstrate the difficulty of making such pronouncements.

\answer

Agreed, and corrected.

\comment

Fig 1: It might be nice to augment this chart with current widely used standard number of bits used in RSA.

\answer

Excellent suggestion.  Done.  Note that this is NIST's 2007
recommendation; RSA appears not to have a single recommendation of
their own at the moment.

\comment

pg. 4 \\
Table 1: ``Josephson junction loop'' $\Rightarrow$ probably better to say ``Superconducting circuit''

\answer

The table has been revised.

\comment

pg. 5\\
This is a great sentence: ``Thus, the principles of classical computer
architecture can be applied, but the answers arrived at will likely
differ substantially from classical architectures.'' and should not be
cut for editing in the article.

\answer

Thank you for the positive comment.

\comment

``Quantum computers do this by making use of the fundamentally quantum
properties of their constituent materials to store and manipulate
data. These are the properties of superposition and entanglement.''

I would restate this in a slightly more restrained manner: while there
are cases that show that small entanglement (defined for example by
schmidt rank of a pure state) implies that the system can be simulated
efficiently clasically, for other more interesting entanglement
measures, this is not always true (in addition the case of mixed state
entanglement is far less settled.)  Somethign like ``Among the novel
quantum properties thought to be important are superposition and
entanglement.)''

\answer

Agreed; the difficulty is conveying this concisely in a lay article.
The sidebar has been substantially revised to focus on gate-based and
adiabatic computation.  The particular point raised by the reviewer
has been addressed.


\comment

``When superposition spreads across more than a single quantum system
and not all elements in the superposition are equally likely, the
qubits can be entangled with each other.''  This feels slightly
misleading because it almost sounds like the only case of ``not''
entanglement is the case of superposition of all equally likely.
Consider rewording.

\answer

Reworded.  Before:

\quotepaper{
When superposition spreads across more than a single quantum system
and not all elements in the superposition are equally likely, the
qubits can be entangled with each other.
}

After:

\quotepaper{
Not all of the elements in the superposition are required to be
equally likely.  In some such cases, when the superposition contains
more than a single qubit, the qubits can be \emph{entangled} with each
other.
}

\comment

pg. 6\\
``The degree to which adiabatic computation is universal is still under
debate.''  Adiabatic quantum computation is certainly universal.  The
open question is whether it is universal under realistic noise
settings.  Maybe choose a slightly different way of expressing this.

\answer

This has been removed as part of the revision of the sidebar.

\comment

pg. 7:\\
``quantum states cannot be duplicated'' should probably be ``unknown quantum states cannot be duplicated''.

\answer

Fixed.

\comment

pg. 7\\
``Probably the most promising approach in error correction is surface
code computation''\\
This seems highly suspect.  First even in quantum architecture I don't
think this is settled dogma.  Second this is definitely not true for
``error correction'' in general: these codes are generally horrible when
it comes to standard measures of quantum error correction.  It is only
when one adds in the constraint of locality that these codes become
interesting.  Finally studies (like those of Svore and the IBM group)
that show other codes performing better in a localized setting.  The
case for surface codes seems to be very much of wide debate.

\answer

Rewritten to be more even-handed.

\comment

pg. 8\\
The second on D-wave needs a big disclaimer stating that their machine
is not known to be universal and performs a specialized class of
quantum algorithms.  I think highlighting their significant
engineering results is very important, and am glad the author does
this, but one still needs to be careful to point out to non-expert
audiences that whether their machine is a full throttled quantum
computer is not known.

\answer

We think the reviewer meant ``the section''.  D-Wave is not mentioned
on p. 8.  The last sentence of the paragraph on D-Wave (formerly on
p. 9, now on p. 4) has been rewritten.  Before:

\quotepaper{
However, they
have yet to definitively demonstrate large-scale entanglement in their
system, or to show that it can actually outperform classical computers
for interesting problems.
}

After:

\quotepaper{
However, the
company has not yet released data that demonstrates coherent quantum
control and entanglement of large numbers of these qubits; as the
device is also not intended to be capable of universal computation, it
is a matter of ongoing debate whether the effects seen are actually
viable as ``quantum computing''.
}

\comment

``and suppress noise''\\
I'm not sure the microwave cavities really suppress noise.

\answer

Deleted in the interest of space.

\comment

pg. 12\\
``(1) a two-level qubit;''\\
This might read many educated CS readers to believe that only
two-level systems have been considered for quantum computers.  Maybe
reword?

\answer

Since DiVincenzo's original criteria specified qubits, we are
reluctant to change this directly, so we have added a footnote.  Note
that this paragraph now starts Section 2 (new numbering).

\section{Reviewer: 2}

Comments to the Author

\comment

Overall, I find this to be a good survey that covers a lot of relevant
recent work in work relating to quantum computer architecture.
My primary concern is that of accessibility to the non-quantum audience.
The sidebar is useful, but I'm concerned that the intro is not well-motivated
without more explanation of quantum computing.  

\answer

The rewritten article has been commented on by several classical
researchers, and the accessibility improved.

The sidebar, rather than being a six-paragraph introduction to all of
quantum computing, now focuses on the two approaches of gate-based
computation and adiabatic computation that appear in the article.

\comment

The discussion of technologies, algorithms and error correction are very
dense, covering many groups.  Perhaps length constraints are a problem
here, but my sense is that more explanation is needed.

\answer

Yes, unfortunately, length is a severe constraint here.  We have
trimmed the number of technologies mentioned directly, in order to
slightly expand the discussion of each.  We hope that motivated readers
will find enough in this review to help them navigate both the more
technical literature and the broad landscape of research groups.

\comment

p9 line 42, multiple groups discussed but only one citation to Chuang's work.
Would be useful to add more citations here

\answer

While we agree, unfortunately, we are already above the recommended
limit of forty references for a CACM review.  Below we propose
thirteen additional references we would like to incorporate into the
article or supplementary online material.

\comment

Shouldn't iARPA be IARPA?  Should there be a citation or link
for the IARPA QCS program?

\answer

IARPA spelling corrected.  Link added as a footnote.

\section{Reviewer: 3}

Comments to the Author

\comment

Overall, the paper is very clear, and definitely reads as a brief,
introductory overview.

\answer

Thank you.

\comment

Technical content is not substantial.

\answer

The purpose of the article is not to present new technical
discoveries, which we believe is what the reviewer refers to here.
This article does present a broad overview of the key technical issues
in scaling up quantum computers, organized in a fashion that we have
not seen before in the literature.  It serves a primary pedagogical
purpose, and secondarily to enumerate areas in which active research
is desirable.

\comment

First, I recommend changing the title.  The current title (and
abstract) do not properly match the article contents.  I recommend a
title on Review of Progress towards a Quantum Architecture, or
something similar. It's really not talking about the state-of-the-art,
but rather overviewing the many subfields involved in quantum
architecture.

\answer

We have opted for the catchier ``How to Design a Quantum Computer''.

\comment

The references are also very limited.  When mentioning algorithms, or
groups, it would be better to also include the reference, instead of
just the group name.  Similarly, when discussing algorithms that show
potential gains over classical approaches, please include the
references to those algorithms and original papers, not just the
review articles.

\answer

While we agree, unfortunately, we are already above the recommended
limit of forty references for a CACM review.  Below we propose
thirteen additional references we would like to incorporate into the
article or supplementary online material.

\comment

More specific comments follow:

p. 1 line 38/39: ``The basic differences'' $\rightarrow$ ``The basic difference between quantum and classical omputers are...''

\answer

The table and this accompanying paragraph have been rewritten.

\comment

p. 1 line 47: ``Fig. 1 demonstrates'' $\rightarrow$ ``Fig. 1 illustrates''  The figure doesn't demonstrate something, it just reviews and illustrates other results.

\answer

Done.

\comment

Why use first names (like Peter Shor) and not just last names?  Same comment applies throughout.

\answer

Corrected.

\comment

p. 2, line 45: ``demonstrates'' $\rightarrow$ illustrates\\
``of making'' $\rightarrow$ ``behind''

\answer

Done.

\comment

p. 4, table coulmn 3: Capitalize all first words in the quantum column.

\answer

Done.

\comment

p. 4, line 36/37: ``to a complete, useful machine''.  I would recommend rewording this.  A complete, useful machine is very generic.  It's to a scalable, real-world machine...

\answer

Done.

\comment

p. 4, line 47: ``engineering-driven''.  It's more that it becomes programmable and engineering-driven.

\answer

Deleted in the process of revising the text.

\comment

p. 5, line 10: ``inside of a'' $\rightarrow$ ``inside a''

\answer

Done.

\comment

p. 5, line 23: Add some references to the algorithms themselves, not just the surveys.

\answer

While we agree, unfortunately, we are already above the recommended
limit of forty references for a CACM review.  Below we propose
thirteen additional references we would like to incorporate into the
article or supplementary online material.

\comment

p. 5, line 27: ``e.g.'' $\rightarrow$ ``e.g.,''

\answer

Done.

\comment

p. 5, line 48/49: ``only n bits of data...'': I recommend phrasing this a little differently, or adding that it's always a probabilistic machine.

\answer

Deleted during revision.

\comment

p.6, line 8 - 10: ``to all the'' $\rightarrow$ ``to all of the''\\
``to get the correct'' $\rightarrow$ ``to get the desired''

\answer

The sidebar has been rewritten.

\comment

p. 6, line 28/29: add ref. to d-wave paper\\

\answer

References to two D-Wave papers are included in the main body of the
article.

\comment

no space before ``-based'' (same error on line 37)

\answer

Fixed.

\comment

p. 7, line 11: ``weights''.  What is meant by weights here?

\answer

Deleted during revision of the QEC section.

\comment

p. 7, line 31: ``will be taken up''.  Reword.  Awkward english.

\answer

Fixed: ``will be taken up correcting errors'' $\rightarrow$ ``will be
used to correct errors''.

\comment

p. 7, line 45: ``highly abstract work''.  I'd remove this.  It makes it
seem hard to grasp, when it's actually had remarkable experimental and
theoretical impact.  Also need to add ref. to the early topological
papers.

\answer

Removed.  Due to tight restrictions on reference count, references to
earlier work (which is foundational but is not expected to be used
directly in implementations) are not included here.

\comment

p. 7 line 49: ``worked tirelessly''.  Again, this is awkward.  I
recommend rewording.

\answer

Done.

\comment

p. 7, line 53: ``Surface codes, of course, are not...'' (there are more
than one, so make plural)

\answer

Deleted during revision.

\comment

p. 8, line 12: change to ``such as the Bacon-Shor code''

\answer

Deleted during revision.

\comment

p. 8, line 49.  Write out Table instead of Tab.

\answer

Done.

\comment

p. 9, line 21: ``leading group''.  You don't say this about any other group, so why have it here?  I'd remove it.

\answer

Done.

\comment

p. 9, line 31: remove ``in journal papers''.  If it's reported, that's assumed to be at a conference or journal.

\answer

Done.

\comment

Harvard architecture?  Please define.

\answer

A Harvard architecture is a machine with separate memories for the
program and data, as implied in the first half of the sentence.  We
believe that the term ``Harvard architecture'' will be commonly
understood by readers of CACM.

\comment

p. 14, line 32: ``e.g.'' $\rightarrow$ ``e.g.,''

\answer

Done.

\section{Reviewer: 4}

Comments to the Author

\comment

This paper focuses on the study of quantum computing
architectures. One of the more important sections, Section 4, does
not, in my opinion, do a very good job of\\
(1) describing the different approaches to quantum information
processing, and, more importantly,\\
(2) their implications on the architectures.

Not only the four criteria that the section opens up with do not
suffice (e.g., what if the system cannot be initialized to a simple
state?---and I suggest discussing DiVincenzo, as well as extended
DiVincenzo criteria here), but the implications those technologies
listed have on the architectures aren't even described.  There does
not appear to be any systematic approach to describing the existing
QIP approaches, and AQC is mixed in the midst of circuit type
approaches.

\answer

After reorganization, this is now Sec. 2 of the paper.  The DiVincenzo
criteria and extended DiVincenzo criteria are discussed in the first
two paragraphs of the section, solving the ``missing fifth criterion''
problem.  Following this, several technologies are introduced in order
of the size of systems built to date.  The impact of the DiVincenzo
criteria on architecture itself appears throughout the text, including
in discussions such as placement of measurement devices here in this
section, at the bottom of the paragraph labeled {\bf heterogeneity}.

\comment

Section 6 uses a number of technical terms without proper definition.

\answer

This section has been deleted and the contents distributed to the
various relevant sections, making it easier to manage new vocabulary
in context.

\comment

Section title announces a list of ``open problems'', however, it
contains a list of research areas.  

\answer

Agreed; the reviewer is correct that the intent is more to illustrate
research areas rather than enumerate a set of constrained, specific
problems.  The contents have been distributed to relevant sections.

\comment

More importantly, I see no connection or natural order in the list of
the areas presented.  It would be helpful to see an overarching
picture, as well as a more structured approach to describing the
individual research areas; possibly, including a more in-depth
description of the areas, too.

\answer

In the originally submitted version, we were trying to create an
implicit flow of first knowing what you are attempting to build
(workloads) and the technologies available, before ending at the
high-level architectural considerations.  However, we now agree with
the reviewer that the article structure was unclear.  More logical
flow is desirable in the paper as a whole, and we have substantially
reorganized it around the layers in Fig. 2 as described in
Sec.~\ref{sec:rearch} of this letter.

\comment

I would, ultimately, like to see some discussions about feasibility of
the different kinds of architectures, as a result of lessons we
learned from QIP, as lessons carried out for the future.  Maybe a
whole separate section?

\answer

For a long list of reasons, none of the large-scale architectures
proposed to date are truly feasible.  The principal lessons learned so
far are included in the rewritten Conclusions, and can be summarized
as:

\begin{itemize}
\item systems capable of solving classically intractable problems will
  be {\em large};
\item device size will limit integration levels, affecting
  architecture; and
\item logical clock speed affects what can and cannot be effectively
  computed (as shown in Figure 2 in the paper).
\end{itemize}

Beyond this, it is difficult to make specific pronouncements, but each
new proposal takes us a step closer to a realizable system.

\comment

Currently, the paper feels very thin in content.  I would, however, be
interested in seeing a well thought through, well structured, and well
laid out survey about quantum architectures from these authors.

\answer

Key sections of the article have been restructured, as described
above.  We also look forward to the opportunity to write a
substantially longer, more detailed survey or tutorial for a different
venue.

\comment

Other comments:\\
Given the controversy regarding D-Wave claims about a quantum
processor/computer, my suggestion is to stay away from mentioning this
company until they prove their device goes beyond what may be achieved
classically.

\answer

With apologies to the reviewer, we believe that, despite the
controversy, inclusion of D-Wave is important at this stage.  See our
discussion of our reasons for including D-Wave in
Sec.~\ref{sec:d-wave} of this letter.

\comment

The reason behind the choice and order of stacks in Figure 2 evades my
understanding. Perhaps it would help if authors explain what is the
meaning of the vertical axis?

\answer

The revised figure includes a legend identifying the layers on the
right side.  The layers move from concrete physical items at the
bottom, through system organization to the abstract reasoning of pure
theory at the top, a structure we feel will be intuitively accessible
to readers.  Note that these layers are not intended to correspond to
or compete with the layers of specific architectural functionality
proposed in the QuDOS paper (Ref.~[18] in the revised article).
Instead, they can be viewed as divisions of human work; a startup
company building a quantum computer would organize the
responsibilities of individuals or teams much along these lines.

\comment

I would order numbers in the long lists of citations in an increasing
order, which seems to be a commonly accepted convention.

\answer

Done.

\comment

I suggest rewriting ``When superposition spreads across more than a
single quantum system and not all elements in the superposition are
equally likely, the qubits can be entangled with each other.'' as ``When
superposition spreads across more than a single basis
(Boolean/register) state, the qubits may be entangled with each
other.''

\answer

Reworded.  Before:

\quotepaper{
When superposition spreads across more than a single quantum system
and not all elements in the superposition are equally likely, the
qubits can be entangled with each other.
}

After:

\quotepaper{
For example, a
3-qubit register can be in the superposed state of all eight values
000 to 111, all with different weights. In some such cases, when the
superposition contains more than a single qubit, the qubits can be
\emph{entangled} with each other. The individual qubits no longer act
independently, and exhibit much more strongly correlated behaviour
than is possible for classical systems.
}

\comment

``Circuit-based quantum computing starts from an equal superposition of
register states'' I am not sure if I agree with that. I believe
circuit-based quantum computing can start with any simple fiducial
state, most often being \ket{0}.

\answer

Many quantum algorithms begin by building the superposition of all
states for a substantial part of the register,
$\ket{0}\rightarrow\sum_{k=0}^{2^n-1}\ket{k}$ (ignoring
normalization), before delving into the heart of the algorithm.  The
details of this process are beyond the scope of this article, but we
have adjusted the text to be more accurate.

Before:

\quotepaper{
Circuit-based quantum computing starts from an equal superposition of
register states and uses gate operations between qubits to change the
weights in the superposition (usually creating entanglement in the
process).
}

After:

\quotepaper{
In general, the first step of a
circuit-based computation is to create an equal superposition of all
register states.  Gate operations between qubits then change the
weights in the superposition, usually creating entanglement in the
process.
}

\comment

I am not sure I agree with the correctness of the description of
AQC. As far as I can tell, AQC algorithm requires a quantum system to
reside in a ground state, whereas a slow enough manufactured evolution
is designed such that by the end of it the ground state is bound to
possess certain desired properties. Furthermore, AQC has been proved
to be equivalent to the circuit computational model with at most
polynomial overhead, http://arxiv.org/abs/quant-ph/0609067

\answer

Rewritten.  New text:

\quotepaper{
The system built by D-Wave is based on a different model of
computation, known as {\em adiabatic quantum computation}. As with the
circuit model, the output state is measured to give the final
answer. In this case, however, the state is designed to be the
low-energy ground state of a quantum system in the quantum
computer. The key to the computation is to adjust the coupling between
quantum systems in the device to allow it to relax into that specific
ground state. The degree to which the D-Wave system achieves this, and
the specific abilities and limitations of the D-Wave system, are
topics of active discussion among the community.
}

\comment

``strict restriction'' is asking to be rewritten.

\answer

Adjective stricken.

\comment

Section 5, in my opinion, mixes algorithm theory and algorithm design,
unable to clearly distinguish between the two.

\answer

In this article, the focus is on how users will evaluate the utility
of quantum computers, and how programmers will create the necessary
applications, rather than the theory behind why quantum algorithms
work.  The section on workloads has been revised and retitled to
reflect this implicit emphasis.


\section{Additional Modifications}

In addition to the requests from the reviewers and editor, the
following modifications have been made:

\begin{itemize}
\item A $10\times$ error in the performance of NFS plotted in Fig. 1
  was discovered and corrected.
\item $n$ replaced with $L$ in Section 5, to be consistent with
  Fig. 1.
\end{itemize}

\section{Additional Proposed References}

The reviewers asked for additional references for work by experimental
groups, the IARPA algorithms, error correction, and adiabatic
computation.  While we have chosen not to write an extensive addendum
to the paper itself to appear as supplementary online material (SOM),
we propose that the references appear either in the article itself or
as a simple list as SOM.  These thirteen additional references would
bring the total reference count of the article to 56, and would
satisfy the reviewers' requests and provide readers with a more
thorough but still accessible reading list.

\begin{thebibliography}{10}

\bibitem{ambainis2007any}
A.~Ambainis, A.~Childs, and B.~Reichardt.
\newblock Any and-or formula of size $n$ can be evaluated in time $n^{1/2+ o
  (1)}$ on a quantum computer.
\newblock In {\em Foundations of Computer Science, 2007. FOCS'07. 48th Annual
  IEEE Symposium on}, pages 363--372. IEEE, 2007.

\bibitem{chiaverini05:qft-impl}
J.~Chiaverini, J.~Britton, D.~Leibfried, E.~Knill, M.~D. Barrett, R.~B.
  Blakestad, W.~M. Itano, J.~D. Jost, C.~Langer, R.~Ozeri, T.~Schaetz, and
  D.~J. Wineland.
\newblock Implementation of the semiclassical quantum {Fourier} transform in a
  scalable system.
\newblock {\em Science}, 308:997--1000, 2005.

\bibitem{childs2003eas}
A.~Childs, R.~Cleve, E.~Deotto, E.~Farhi, S.~Gutmann, and D.~Spielman.
\newblock {Exponential algorithmic speedup by a quantum walk}.
\newblock In {\em Proceedings of the thirty-fifth annual ACM symposium on
  Theory of computing}, pages 59--68. ACM New York, NY, USA, 2003.

\bibitem{dennis:topo-memory}
E.~Dennis, A.~Kitaev, A.~Landahl, and J.~Preskill.
\newblock Topological quantum memory.
\newblock {\em J. Math. Phys.}, 43:4452--4505, 2002.

\bibitem{hallgren2007ptq}
S.~Hallgren.
\newblock {Polynomial-time quantum algorithms for Pell's equation and the
  principal ideal problem}.
\newblock {\em Journal of the ACM (JACM)}, 54(1), 2007.

\bibitem{harrow:lineqs}
A.~W. Harrow, A.~Hassidim, and S.~Lloyd.
\newblock Quantum algorithm for linear systems of equations.
\newblock {\em Phys. Rev. Lett.}, 103(15):150502, Oct 2009.

\bibitem{hensinger06:_t-junction}
W.~K. Hensinger, S.~Olmschenk, D.~Stick, D.~Hucul, M.~Yeo, M.~Acton,
  L.~Deslauriers, and C.~Monroe.
\newblock {T}-junction ion trap array for two-dimensional ion shuttling,
  storage, and manipulation.
\newblock {\em Applied Physics Letters}, 88:034101, 2006.

\bibitem{magniez2005quantum}
F.~Magniez, M.~Santha, and M.~Szegedy.
\newblock Quantum algorithms for the triangle problem.
\newblock In {\em Proceedings of the sixteenth annual ACM-SIAM symposium on
  Discrete algorithms}, pages 1109--1117. Society for Industrial and Applied
  Mathematics, 2005.

\bibitem{PhysRevLett.99.070502}
A.~Mizel, D.~A. Lidar, and M.~Mitchell.
\newblock Simple proof of equivalence between adiabatic quantum computation and
  the circuit model.
\newblock {\em Phys. Rev. Lett.}, 99:070502, Aug 2007.

\bibitem{regev2002quantum}
O.~Regev.
\newblock Quantum computation and lattice problems.
\newblock In {\em Foundations of Computer Science, 2002. Proceedings. The 43rd
  Annual IEEE Symposium on}, pages 520--529. IEEE, 2002.

\bibitem{svore05:_local-ft}
K.~M. Svore, B.~M. Terhal, and D.~P. DiVincenzo.
\newblock Local fault-tolerant quantum computation.
\newblock {\em Physical Review A}, 72:022317, 2005.

\bibitem{takakura2010triple}
T.~Takakura, M.~Pioro-Ladri{\`e}re, T.~Obata, Y.~Shin, R.~Brunner, K.~Yoshida,
  T.~Taniyama, and S.~Tarucha.
\newblock Triple quantum dot device designed for three spin qubits.
\newblock {\em Applied Physics Letters}, 97(21):212104, 2010.

\bibitem{whitfield2011simulation}
J.~Whitfield, J.~Biamonte, and A.~Aspuru-Guzik.
\newblock Simulation of electronic structure hamiltonians using quantum
  computers.
\newblock {\em Molecular Physics}, 109(5):735--750, 2011.

\end{thebibliography}

\end{document}
